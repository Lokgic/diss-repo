\chapter{Conclusion}

Peirce's Pragmatic Maxim and tripartite classification of different contexts of inquiry provide a unified way of understanding how epistemic commitments figure into our probabilistic inferences and experimental procedures. Making inductive inferences based on experimental procedures, I argued, presupposes that we can rely on the experiment to behave deliberately according to the setup of the experiment promised at the outset. If the experiment is stopped earlier or later than promised, the character of the inference could be drastically changed. 

How inferences could be dependent on whether these practical commitments are satisfied is the crucial question left open in chapter 2. This is addressed in chapter 3: my focus of this chapter was to elucidate the very idea that the commitment made in the abductive context has repercussions on the investigator's deliberate conduct during the inductive stage of inquiry. The problem of optional stopping in parapsychology served as a helpful case to demonstrate the issues at stake. Even though this problem has been traditionally associated with frequentist methodologies, I have tried to show that it should also concern Bayesians by reproducing a similar result using statistical simulations. 

My contention is that aspects of inductive reasoning have to be criticized in light of the deliberation the inquirer undergoes prior to the experiment, such as the commitments and intentions considered and decided by the experimenter, all of which directly impact the statistical model used for inductive inference. Error probabilities are especially important in the evaluation of the experimenter's intention to allow her hypothesis to be confronted by experience in a fair manner.

 I explored a Bayesian response to optional stopping, which relies on a mathematical result, proven by Ramsey and others, that one's expected value never decreases from gathering new evidence, and in many cases it actually increases. I dismissed this line of thought, because the result only holds when the evidence is cost-free, which is almost never the case.
 
The central concern of chapter 4 was the dynamic between the abductive and deductive contexts of inquiry. We saw that Peirce makes the tantalizing suggestion that abduction is \emph{interrogative}. To explain what this means, I borrowed Hintikka's helpful connection between the economics of choosing a hypothesis for inductive testing and choosing an assumption for premiseless proofs. Both involve a strategic element in thinking about the deductive implication of what \emph{would} happen, had the choice been made a certain way. I tried to apply this insight on Keynes' idea of the weight of evidence, and suggested that the evidential weight represents a \emph{dispositional commitment}, revealed through a deductive probability calculus from the chosen prior. We saw that in the case of Popper's paradox of ideal evidence, what has been changed is not the posterior probability, but our willingness to revise our belief in light of new evidence.  

